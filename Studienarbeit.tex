%%% Standardeinstellungen für die Studienarbeit
\documentclass[a4paper,12pt]{scrreprt}
\usepackage[onehalfspacing]{setspace}
\usepackage[left=4cm,right=2cm,top=2cm,bottom=1cm,footnotesep=1.5em,includeheadfoot]{geometry} %bei Heftung/Bindung hier: left=5.5cm
\usepackage{url} %URLs anklickbar, aber nur im PDF nicht im Druck

%%%%%%%%% Abkürzungen %%%%%%%%%
%optional nützlich
\usepackage{xspace}
\xspaceaddexceptions{“}
\newcommand{\gap}{[…]\xspace}
\newcommand{\zb}{z.~B.\xspace}
\newcommand{\ZB}{Z.~B.\xspace}
\newcommand{\dah}{d.~h.\xspace}
\newcommand{\me}{m.~E.\xspace}
\newcommand{\su}{s.~u.\xspace}
\newcommand{\ua}{u.~a.\xspace}

%%%%%%%%%% Sprachen %%%%%%%%%%%%%
\usepackage{hyperref} %Literatur und Inhaltsverzeichnis  anklickbar, aber nur im PDF nicht im Druck
\usepackage[german=quotes]{csquotes}
\usepackage{polyglossia} %geht nur mit XeLaTeX-Compiler
\setmainfont{Times New Roman}
\setdefaultlanguage[spelling=new,babelshorthands]{german}
\setotherlanguage[variant=american]{english}
\setotherlanguage[variant=ancient]{greek}
\newfontfamily\greekfont{SBL Greek} %vorher https://www.sbl-site.org/educational/BiblicalFonts_SBLGreek.aspx installieren
\setotherlanguage{hebrew} %vorher https://www.sbl-site.org/educational/BiblicalFonts_SBLHebrew.aspx installieren
\newfontfamily\hebrewfont[Script=Hebrew,Contextuals=Alternate]{SBL Hebrew}
\newcommand{\heb}[1]{\texthebrew{#1}}
%\setsansfont{Liberation Sans} % or some other sf font with Hebrew glyphs
%\newcommand{\captheb}[1]{\textrm{\texthebrew{#1}}}
\DeclareRobustCommand*{\captheb}[1]{\textrm{\texthebrew{#1}}}
\newcommand{\grk}[1]{\textgreek{#1}}

\newcommand{\hsd}{\heb{חֶ֫סֶד}\xspace}
\newcommand{\amt}{\heb{אֱמֶת}\xspace}
\newcommand{\hn}{\heb{חֵן}\xspace}
\newcommand{\rhmm}{\heb{רַחֲמִים}\xspace}

%%%%%%%%%% Bibliographie %%%%%%%%%%%%%%
%Es hilft http://tug.ctan.org/info/biblatex-cheatsheet/biblatex-cheatsheet.pdf
\usepackage{xpatch}
\usepackage[style=verbose-ibid,citestyle=authortitle-ibid,hyperref=true,backend=biber,sortlocale=de,language=german,mincrossrefs=1000]{biblatex}
\addbibresource{lit.bib} %{../../literatur/lit.bib} 
%empfehlenswertes Programm jabref, der Zitierstil @online muss hier noch ausgefüllt werden
%empfehlenswert vorausgefüllte Eigenschaften zu einer Quelle aus https://scholar.google.de/schhp?hl=de oder Google-Books importieren | Die Datei ist empfehlenswert in einem höheren Verzeichnis aufzubewahren, um sie für alle Studienarbeiten nutzen zu können

\DeclareFieldFormat{pages}{#1}	% keine "S."-Angaben
\DeclareFieldFormat{postnote}{#1}	% keine "S."-Angaben
\renewcommand{\newunitpunct}{\addcomma\space}

\DeclareBibliographyDriver{incollection}{%
  \usebibmacro{bibindex}%
  \usebibmacro{begentry}%
  \usebibmacro{author/translator+others}%
  \setunit{\labelnamepunct}\newblock
  \usebibmacro{title}%
  \newunit
  \printlist{language}%
  \newunit\newblock
  \usebibmacro{byauthor}%
  \newunit\newblock
  \usebibmacro{in:}%
  \usebibmacro{byeditor+others}%
  \newunit\newblock
  \usebibmacro{maintitle+booktitle}%
  \newunit\newblock
  \printfield{edition}%
  \newunit
  \iffieldundef{maintitle}
    {\printfield{volume}%
     \printfield{part}}
    {}%
  \newunit
  \printfield{volumes}%
  \newunit\newblock
  \usebibmacro{series+number}%
  \newunit\newblock
  \printfield{note}%
  \newunit\newblock
  \usebibmacro{publisher+location+date}%
  \newunit\newblock
  \usebibmacro{chapter+pages}%
  \newunit\newblock
  \iftoggle{bbx:isbn}
    {\printfield{isbn}}
    {}%
  \newunit\newblock
  \usebibmacro{doi+eprint+url}%
  \newunit\newblock
  \usebibmacro{addendum+pubstate}%
  \setunit{\bibpagerefpunct}\newblock
  \usebibmacro{pageref}%
  \newunit\newblock
  \iftoggle{bbx:related}
    {\usebibmacro{related:init}%
     \usebibmacro{related}}
    {}%
  \usebibmacro{finentry}}

\renewbibmacro*{byeditor+others}{% %Hg. bzw Hgg. in Klammern hinter Herausgeber, statt davor
        \ifnameundef{editor}
        {}
        {\printnames[byeditor]{editor}%
                \setunit{\addspace}%
                \usebibmacro{editor+othersstrg}%
                \clearname{editor}%
                \newunit}%
        \usebibmacro{byeditorx}%
        \usebibmacro{bytranslator+others}}

\DefineBibliographyStrings{german}{%
        andothers = {et\ al\adddot}, %et al. statt u.a.
        editor = {\mkbibparens{Hrsg\adddot}}, %Hg. statt Hrsg.
        editors = {\mkbibparens{Hrsg\adddot}}, %Hgg. plural
        byeditor = {\mkbibparens {Hrsg\adddot}},
}

\xpatchbibmacro{editor}%
  {\setunit{\addcomma\space}}% no comme before editor string ...
  {\setunit{\addspace}}% ... just a space
  {}
  {}
 \xpatchbibmacro{bbx:editor}% some styles use a fancier version of the macro
  {\setunit{\addcomma\space}}% no comme before editor string ...
  {\setunit{\addspace}}% ... just a space
  {}
  {}


%Beispiele Unterkategorien in Bibliothek zu bilden
%\DeclareBibliographyCategory{hesed}
%\addtocategory{hesed}{David1000}
%\nocite{Freude2000}
%\DeclareBibliographyCategory{hen}
%\addtocategory{hen}{Danke2001,Bitten2002}
%\DeclareBibliographyCategory{rhm}
%\addtocategory{rhm}{...}
%\DeclareBibliographyCategory{amn}
%\addtocategory{amn}{...}

\newcommand{\at}[1]{\textsc{#1}}

% Keine "Schusterjungen"
\clubpenalty = 1000
% Keine "Hurenkinder"
\widowpenalty = 1000 \displaywidowpenalty = 1000
\raggedbottom % Sorgt beim Dokumenttyp book dafür, daß kein Ausgleich des unteren Seitenrandes durch Dehnung der Absatzabstände durchgeführt wird. \raggedbottom ist bei den Dokumenttypen article, report und letter bereits voreingestellt. 

\begin{hyphenrules}{ngerman}
\hyphenation{%
text-ex-ter-ne
Glueck
}
\end{hyphenrules}


\begin{document}
\pagenumbering{arabic}
\addtocounter{tocdepth}{1}
\addtocounter{secnumdepth}{1}

%%%%%%%%% Titel %%%%%%%%%%%%%

%\titlehead{ }
%\subject{ }
\title{\MakeUppercase{Freut euch im Herrn allezeit, abermals sage ich: Freut euch!}} %\MakeUppercase macht Großsschreibung
\subtitle{\MakeUppercase {Eine Einladung zur Freude am Herrn, nach Philipper 4,4}}
\author{Seminarbeit für die Lehrveranstaltung\\
Basismodul PT (PT I):\\
Grundlagen der Vermittlung des Evangeliums\\ %ggf. erst Lehrveranstaltung innerhalb des Moduls und darunter dann das übergeordnete Modul
Freie Theologische Hochschule Gießen\\
2. Semester, B.A.\\
Prof. Dr. Philipp Bartholomä}
\date{ } %leer bleiben, da fehlplatziert

\publishers{Florian Kurrle\\ \today} %Bei BA/MA mit Ort
\maketitle

%%%%%% nur bei BA/MA notwendig %%%%%%%%%%%%%%
%\thispagestyle{empty}
%\section*{Eigenständigkeitserklärung}
%Hiermit erkläre ich wahrheitsgemäß, dass ich diese Arbeit selbstständig und inhaltlich ohne fremde Hilfe verfasst und keine anderen als die angegebenen Hilfsmittel benutzt habe. Die Stellen dieser Arbeit, die anderen Quellen im Wortlaut oder dem Sinn nach entnommen wurden, sind durch Angaben der Herkunft im Einzelnen kenntlich gemacht. Dies gilt auch für Zeichnungen, Skizzen, bildliche Darstellungen sowie für Quellen aus dem Internet. Die vorliegende Arbeit ist noch nicht veröffentlicht oder in gleicher oder anderer Form an irgendeiner Stelle als Prüfungsleistung vorgelegt worden.
%Ort, \today Unterschrift 
%\newpage

\tableofcontents
\pagestyle{headings}

%%%%%% Inhalt %%%%%%%%%%%%%%

\chapter{\label{Forschung}Forschungsüberblick}
\section{\label{kognitiv}Kognitive Freude}

\section{\label{valenz}Dependenz und Valenz}

\section{\label{case}Kasusknacksus}

\begin{itemize}
\item \heb{יהוה} ist der \textit{Agentiv} schön \footcite[Buch][11ff.]{Alter2011}
\end{itemize}

Diese semantische Information \footcite[nach (Zeitschrift-)Artikel][8-12]{Williamson2009}

\begin{quote}
wird \footcite[nach Lexika][8-12]{Khan2013} ausgeübt.\footcite[nach Sammelband][8-12]{Fillmore1968}
\end{quote}
\section{\label{Semantik}Semantik}

Die Startseite der FTH \footcite{FTHStartseite2019}

% % %Beispiel wenn bei Bachelor-Thesis oder Master-Thesis mit separaten Dateien gearbeitet wird
%\input{prob}
%\input{for}
%\input{meth}
%\input{hesed}
%%\input{hen}
%%\input{erg}

%%%%%%% Bibliographie %%%%%%%%%%
%Es hilft http://tug.ctan.org/info/biblatex-cheatsheet/biblatex-cheatsheet.pdf

\sloppy %soll Worttrennungen vermeiden

\printbibliography [heading=bibintoc]
\label{Literatur}

%Beispiele Unterkategorien in Bibliothek zu bilden
%%\printbibliography[heading=bibintoc]
%%\printbibliography[category=hesed,heading=subbibliography,title={Zu \captheb{חֶ֫סֶד} noch lesen:}]
%%\printbibliography[category=hen,heading=subbibliography,title={Zu \captheb{חֵן} auswerten:}]\printbibliography[category=rhm,heading=subbibliography,title={Zu \captheb{רחם} auswerten:}]
%%\printbibliography[category=amn,heading=subbibliography,title={Zu \captheb{אמן} auswerten:}]

\end{document}